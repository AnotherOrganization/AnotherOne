Course Management subsystem

The course management subsystem is responsible for any operations that
are related to the courses as well as its impact on the schedule itself.
Each section inherits a course to know all its information. Each course
has at least one section. Also each course may have prerequisites which
have to be met in order to be able to add it to the schedule. Each
section for a specific course may have a tutorial and a laboratory
component. Note that each section, tutorial and laboratory all have one
or more room blocks. This is necessary because each of those classes may
have different time and room allocations during the week. The roomBlock
class inherits the timeBlock class to know at which time and during
which semester the room is going to be occupied. The timeBlock class
inherits the preferences to know which times are preferred by the user.
The current schedule may have different sections for which the user is
already registered to.

\textbf{\emph{Schedule:}} is a views class that provides data about the
courses the user wants to add.

Attributes:

\begin{itemize}
\item
  N/a
\end{itemize}

Methods:

\begin{itemize}
\item
  addSection(s : section) will temporarily add the section s, which is
  the parameter, to the schedule.
\item
  removeSection(s : section) will remove the section s, which is the
  parameter, from the schedule.
\item
  addTutorial(Tutorial) will add the tutorial that is associated to the
  section (if any)
\item
  addLaboratory(Laboratory) will add the laboratory that is associated
  to the section (if any)
\end{itemize}

\textbf{\emph{Section:}} is a views class for a specific course.
Depending on the course, it may have one or more sections.

Attributes:

\begin{itemize}
\item
  instructor: string
\end{itemize}

Methods:

\begin{itemize}
\item
  getTutorials(): Tutorials{[}*{]} will return an array of tutorials for
  a specific section of a course.
\item
  getLaboratories(): Laboratories{[}*{]} will return an array of
  laboratories for a specific section of a course.
\end{itemize}

\textbf{\emph{Tutorial:}} is a views class which represents the tutorial
for a specific section of a course (if applicable.

Attributes:

\begin{itemize}
\item
  TA: string
\end{itemize}

Methods:

\begin{itemize}
\item
  N/A
\end{itemize}

\textbf{\emph{Laboratory:}} is a views class which represents the
laboratory for a specific section of a course (if applicable).

Attributes:

\begin{itemize}
\item
  Lab\_instructor: string
\end{itemize}

Methods:

\begin{itemize}
\item
  N/A
\end{itemize}

RoomBlock: is a views class which represents the room(s) in which a
class is taking place.

Attributes:

\begin{itemize}
\item
  id: string
\item
  room: string
\end{itemize}

Methods:

\begin{itemize}
\item
  N/A
\end{itemize}

TimeBlock: is a views class that represents the time for which the room
is needed.

Attributes:

\begin{itemize}
\item
  start: Date
\item
  end: Date
\item
  semester: int
\end{itemize}

Methods:

\begin{itemize}
\item
  overlaps(TimeBlock timeBlock): bool will look at timeBlock (which is
  the parameter) and checks if there is any conflict in terms of time.
  If there is a conflict, then the method will return true, otherwise
  false.
\end{itemize}

Preference: is a views class which represents the user's desired options
when generating a schedule.

Attributes:

\begin{itemize}
\item
  N/A
\end{itemize}

Methods:

\begin{itemize}
\item
  N/A
\end{itemize}

Course: is a views class which represents a subject that needs to be
completed in order to graduate. It has at least one section.

Attributes:

\begin{itemize}
\item
  name: string
\item
  subject: string
\item
  courseNumber: int
\item
  credits: double
\item
  courseDescription: string
\end{itemize}

Methods:

\begin{itemize}
\item
  getSections(): Section{[}*{]} will return an array of different
  sections for a specific course.
\item
  isAvailable(string semester): bool will take a string semester as
  parameter and check if the course is available for that semester. If
  it is available, the method will return true, otherwise it will return
  false.
\item
  getPrerequisites(): Course{[}*{]} will return an array of different
  courses that are prerequisites for a specific course.
\item
  arePrerequisitesMet(): bool will check if the prerequisites are met
  for a specific course. If the prerequisites are met, then the method
  will return true, else it will return false.
\end{itemize}

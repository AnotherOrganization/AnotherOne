\emph{Activity Diagram Description:}

Activity diagrams are used to display the process view. The activity
diagram, describes precisely the order of the student's activities
during registration. The activity diagram is similar to a flow chart of
activities ,and that demonstrates the capability of the system, as well
as the results when errors may occur caused by the users' decisions.

In the Activity Diagram provided, it shows the system activity path from
login until final registration.

After the user has successfully logged in, the user may decide day and
time preferences for their schedule. After selecting the time and day
constraints, a list of courses based on the constraints will appear on
the screen and the user may decide the courses they wish to be
registered in.

Following this, the user may decide an auto selection or manual
selection of their courses. If Auto selection is selected, the courses
available based on their pre-requisites will be generated into the final
schedule. The user may then modify these selection, if they would like.
The next option is manual selection, where the user will enter the
course number into the search box and choose a section. This selection
will then go to check if the user has the prerequisite. If a required
prerequisite is not satisfied,

, a message will appear and the user must choose another course. The
user may drop their course, or modify the courses they chose at any
given moment during their registration period. If the prerequisite is
satisfied, the course will be added to the student's schedule.

Finally, the user just needs to commit their schedule and the final
schedule is generated. The user may then proceed to log out or redo
their schedule completely as all of this activity occurs on one page.

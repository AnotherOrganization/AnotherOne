AUTHENTICATION SUBSYTEM

\includegraphics[width=6.50000in,height=6.42777in]{media/image1.jpeg}

4.1

The authentication subsystem is responsible for validating user logins
as well as providing security throughout the entire session and hashing
sensitive information such as passwords. The user class is the starting
point of the authentication subsystem, it is found within the models
therefore it must inherit from the template class CI\_Model. The user
begins by entering their username and password in order to login to
MyTinerary. This in turn calls the authenticate method which will take
in the entered login name and password and verify them by querying the
database. If the query results in one row and the internal password
verify method returns true than the login is successful. During this
time the User class will interact with the Login Controller class. If
the authenticate method is successful the Login class will redirect the
user to the index page using the index method as well as providing the
user with a new session. The CI\_Security is a singleton class that is
the cornerstone of MyTinerary's secure session handling. It creates a
unique hashed Cross Site Request Forgery protection cookie for every
user session. MyTinerary also uses the BlowFish algorithm within the
database to hash all sensitive data such as passwords or course details.

\textbf{4.2}

The following two classes are provided by the CodeIgniter framework that
allow for a separation of concerns via the MVC pattern which promotes
high cohesion and low coupling by allowing each module to work
independently of one another. Please note these classes do not need to
be defined but simply inherited from.

\textbf{\emph{CI\_Model:}} Is a template class provided by CodeIgniter
in which any class containing the domain logic will inherit from.

Attributes:

\begin{itemize}
\item
  N/A
\end{itemize}

Methods:

\begin{itemize}
\item
  \_\_construct(): Class constructor initializing a default log message.
\item
  \_\_get(key): Debugging method that returns an error message.
\end{itemize}

\textbf{\emph{CI\_Controller:}} Is a template class provided by
CodeIgniter in which any class that manipulates the user interface
classes will inherit from.

Attributes:

\begin{itemize}
\item
  N/A
\end{itemize}

Methods:

\begin{itemize}
\item
  \_\_constructor(): Class constructor loading respective view classes.
\item
  get\_instance(): Returns controller instance.
\end{itemize}

\textbf{\emph{Login}:} Is a controller class that redirects the user
according the result of the User's authentication method.

Attributes:

\begin{itemize}
\item
  N/A
\end{itemize}

Methods:

\begin{itemize}
\item
  \_\_construct(): Class constructor
\item
  index(): Validates if authentication method has succeeded, sets
  session cookies, and redirects user to home page. Returns void.
\item
  signout(): Destroys current user session and redirects to default
  login page. Returns void.
\end{itemize}

\textbf{\emph{User}:} Is a model class that directly interacts with the
database containing all user information.

Attributes:

\begin{itemize}
\item
  login\_name: string
\item
  password: string
\item
  first\_name: string
\item
  last\_name: string
\item
  email: string
\end{itemize}

Methods:

\begin{itemize}
\item
  \_\_construct(): Class constructor calling the parent constructor.
\item
  authenticate(login\_name,password): This method takes in a two strings
  being the login name and password and queries the database to find a
  match. Note the strings are immediately hashed using the blowfish
  algorithm to match with the database elements. Returns a boolean.
\item
  is\_admin(user\_id): Checks if user is an administrator returning a
  Boolean used to grant different access rights compared to a regular
  student.
\item
  get\_user\_info(user\_id): Returns user information from the database
  such as login name and email. Returns user information in the form of
  an array of strings.
\end{itemize}

\textbf{\emph{CI\_Security:}} Is a singleton class provided by
CodeIgniter in which every component of the MVC classes will use. The
CI\_Security class contains many attributes and methods however only a
select few are relevant to the authentication subsystem. The
CI\_Security provides protection for sensitive information as well as
secure session handling.

Attributes:

\begin{itemize}
\item
  \_xss\_hash: Randomly generated hash for protecting MyTinerary's URLs.
\item
  \_csrf\_hash: Randomly generated hash for Cross Site Request Forgery
  protection cookie.
\item
  \_csrf\_token\_name: Token name for Cross Site Request Forgery
  protection cookie.
\item
  \_csrf\_cookie\_name: Cookie name for Cross Site Request Forgery
  protection cookie.
\end{itemize}

The xss and csrf hash allows for the protection of user sessions since
the hash is randomly generated and provides secure access using either
the token or cookie name. Please note that the usage of this class is
all done internally within the CodeIgniter framework.

Methods:

\begin{itemize}
\item
  \_\_construct(): Class constructor that will create the cookies and
  set the hash
\item
  csrf\_verify(): Provides security validation on POST requests,
  validates if URI has been whitelisted, as well as garbage collection
  on finished session arrays. Returns the session if verified otherwise
  returns null.
\item
  csrf\_set\_cookie(): Modifies the internal cookie and returns it.
  Returns void.
\item
  csrf\_show\_error(): Displays an error message when user of
  application attempts an action that is either not permitted or not
  secure. Returns void.
\item
  get\_csrf\_hash(): Return the hash as a string.
\end{itemize}
